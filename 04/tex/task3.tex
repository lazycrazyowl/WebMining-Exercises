\begin{frame}[c]
	\myframetitle{Task 3 - Compress}{Anforderungen an das Kompressionsverfahren}
\begin{itemize}
  \fatitem{Klassifikation erfolgt via Dateigröße} 
  \item Das Dokument d wird zu jeder einzelnen Klasse mit Trainingsdaten
  hinzugefügt und komprimiert.
  \item d gehört dann zur Klasse mit der kleinsten Dateigrößenänderung durch das
  Hinzufügen von d.
  \item Daraus ergeben sich folgende Anforderungen an das Kompressionsverfahren:
  \begin{itemize}
    \item Ähnliche Dokumente müssen stark komprimiert werden
    \item Unähnliche Dokumente müssen schwach komprimiert werden
  \end{itemize}
\end{itemize}
\end{frame}

\begin{frame}[c]
	\myframetitle{Task 3 - Compress}{Mögliche Kompressionsverfahren:}
\begin{itemize}
  \fatitem{ Mögliche Kompressionsverfahren:}
\item Zur Evaluation der Kompressionsverfahren wurde WebKb.tar komprimiert
  \begin{table}
\begin{tabular}{|c|c|c|c|c|}
\hline
Verfahren & Blockweise & Coding & CPU-Kosten & Kompression\\
\hline
GZip & nein & Entropie & 19s & 10MB \\
BZip2 & ja & Entropie & 42s & 8MB \\
LZMA & nein & Dictionary & 132s & 7,5MB \\
\hline
\end{tabular}
\caption{Kompressionsverfahren im Vergleich}
\label{tbl:CompressionRatio}
\end{table}
\item LZMA liefert vermutlich die besten Ergebnisse
\begin{itemize}
  \item Wegen hoher Kompression
  \item Und der Verwendung eines Dictionaries
\end{itemize}
\item Aber LZMA benötigt zu viel Rechenzeit
\end{itemize}

\end{frame}

\begin{frame}[c]
\myframetitle{Task 3 - Compress}{Ablauf der Kompression}
\begin{itemize}
  \item Zur Kompression wird GZip verwendet
  \item Es sollen die GZip-Kompressionsstufe 1, 2, 6 und 9 verglichen werden
  \fatitem{Die Klassifikation erfolgt in folgenden Schritten}
  \begin{enumerate}
  \item Vorbereitung: Zusammenfassung der Trainingsdaten jeder Klasse zu
  \keyword{class.tar}
  \item Kompression mit gzip von \keyword{class.tar} zu \keyword{class.tar.gz}
  \item Erzeugen von \keyword{d.class.tar.gz} durch Hinzufügen des zu
  klassifizierenden Dokuments d zu \keyword{class.tar.gz}
  \item Klassifikation von d duch die Formel
  $min(len(d.class.tar.gz)-len(class.tar.gz))$
  \item Speicherung des Klassifikiationsergebnisses in einer Ergebnismatrix
\end{enumerate}
\end{itemize}
\end{frame}

