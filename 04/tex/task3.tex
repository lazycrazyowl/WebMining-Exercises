\begin{frame}[c]
	\myframetitle{Task 3 - Compress}{Anforderungen an das Kompressionsverfahren}
\begin{itemize}
  \fatitem{Klassifikation erfolgt via Dateigröße} 
  \begin{itemize}
    \item Ähnliche Dokumente müssen stark komprimiert werden
    \item Unähnliche Dokumente müssen schwach komprimiert werden
  \end{itemize}
  \item Annahme: Blockweise Kompressionsverfahren erfüllen unsere Anforderungen
  nicht bestmöglich.
\end{itemize}
\end{frame}

\begin{frame}[c]
	\myframetitle{Task 3 - Compress}{Mögliche Kompressionsverfahren:}
\begin{itemize}
  \fatitem{ Mögliche Kompressionsverfahren:}
  \begin{table}
\begin{tabular}{|c|c|c|c|c|}
\hline
Verfahren & Blockweise & Coding & CPU-Kosten & Kompression\\
\hline
BZip2 & ja & Entropie & mittel & gering \\
GZip & nein & Entropie & gering & mittel \\
LZMA & nein & Dictionary & hoch & hoch \\
\hline
\end{tabular}
\caption{Kompressionsverfahren im Vergleich}
\label{tbl:CompressionRatio}
\end{table}
\item Rechenzeit ist für die Evaluation vermutlich nebensächlich
\item LZMA ist für die Textklassifikation vermutlich am besten geeignet
\end{itemize}

\end{frame}

\begin{frame}[c]
\myframetitle{Task 3 - Compress}{Ablauf der Kompression}

\end{frame}

