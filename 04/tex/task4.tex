\begin{frame}[c]
	\myframetitle{Task4}{Evaluation von Naive Bayes}
	\begin{itemize}
	\fatitem{Klassifikationsmatrix für Naive Bayes Algorithm mit alles Features}	
	
	\begin{table}[htbp]
		\begin{tabular}{|l|r|r|r|r|r|r|r|}
		\hline
		train$\downarrow$ test$\rightarrow$& project &
		course & other & student & faculty & dpt. & staff \\ \hline
		
		project & 0 & 0 & 252 & 0 & 0 & 0 & 0 \\ \hline
		course  & 0 & 0 & 465 & 0 & 0 & 0 & 0 \\ \hline
		other   & 0 & 0 & 1880 & 0 & 1 & 0 & 0 \\ \hline
		student & 0 & 0 & 821 & 0 & 0 & 0 & 0 \\ \hline
		faculty & 0 & 0 & 561 & 0 & 1 & 0 & 0 \\ \hline
		dpt.    & 0 & 0 & 91 & 0 & 0 & 0 & 0 \\ \hline
		staff   & 0 & 0 & 68 & 0 & 0 & 0 & 0 \\ \hline
	\end{tabular}
	\caption{Resultate Naive Bayes}
	\end{table}
	
	\item Alle Dokumente werden als \textit{other} kategorisiert => Overfitting
	
	\end{itemize}
\end{frame}

\begin{frame}[c]
	\myframetitle{Task4}{Evaluation von Naive Bayes}
	\begin{itemize}
	\fatitem{Klassifikationsmatrix für Naive Bayes Algorithm mit 50,000 Features}	
	
	\begin{table}[htbp]
		\begin{tabular}{|l|r|r|r|r|r|r|r|}
		\hline
		train$\downarrow$ test$\rightarrow$& course &
		student & project & dpt. & staff & other & faculty \\ \hline
		
		course  & 12 & 18 & 9 & 1 & 0 & 394 & 31 \\ \hline
		student & 4 & 44 & 10 & 0 & 0 & 714 & 49 \\ \hline
		project & 4 & 9 & 6 & 0 & 1 & 215 & 17 \\ \hline
		dpt.    & 0 & 2 & 0 & 0 & 0 & 86 & 3 \\ \hline
		staff   & 0 & 3 & 1 & 0 & 0 & 60 & 4 \\ \hline
		other   & 9 & 91 & 20 & 3 & 3 & 1660 & 95 \\ \hline
		faculty & 6 & 26 & 6 & 0 & 1 & 481 & 42 \\ \hline
	\end{tabular}
	\caption{Resultate Naive Bayes}
	\end{table}
	
	\item Weniger Overfitting. Die meisten Dokumente sind aber immer noch in der Kategorie \textit{Other}
	
	\end{itemize}
\end{frame}

\begin{frame}[c]
	\myframetitle{Task4}{Recall, Precision, Accuracy von Naive Bayes}
	\begin{itemize}
	\fatitem{Auswertung der Klassifikationsmatrizen ergibt}
	
	\begin{table}[htbp]
		\begin{tabular}{|l|r|r|r|r|}
		\hline
		\textbf{Test} & \textbf{NB(93.311 Feat)} & \textbf{NB(5.,000 Feat)} \\ \hline
		Accuracy & 45,43\% & 42,61\% \\ \hline

		\multicolumn{3}{|l|}{\textbf{Recall}}  \\ \hline
		micro & 55,43\% & 38,74\% \\ \hline
		macro & 14,34\% & 16,79\% \\ \hline
		
		\multicolumn{3}{|l|}{\textbf{Precision}}  \\ \hline
		micro & 100\% & 89,36\% \\ \hline
		macro & NaN\% & 38,11\% \\ \hline
		\end{tabular}
		
		\caption{Auswertung der Klassifikationsresultate für Naive Bayes}
	\end{table}
  
	\end{itemize}
\end{frame}

\begin{frame}[c]
	\myframetitle{Task4}{Interpretation der Ergebnisse}
	\begin{itemize}
	  \item Die Ergebnisse erreichen gerade mal die Baseline (raten)
	  \item Nutzung aller Token resultiert in Overfitting
	  \item Reduzierung der Tokens bringt aber keine Verbesserung
	  \item Problem:
	  \begin{enumerate}
	    \item WebKB besteht aus vielen kleinen Texten
	    \item Zu wenig Text um Ähnlichkeiten Festzustellen
	    \item Datei in der selben Kategorie sind ziemlich unterschiedlich
	    \item Kaum Worte die in mehreren Texten vorkommen
	  \end{enumerate}
	\end{itemize}
\end{frame}

\begin{frame}[c]
	\myframetitle{Task 4}{Evaluation von GZip}
\begin{itemize}
  \fatitem{Klassifikationsmatrix für GZip(Kompressionsstufe 1)}
 \begin{table}[htbp]
\begin{tabular}{|l|r|r|r|r|r|r|r|}
\hline
train$\downarrow$ test$\rightarrow$& \multicolumn{1}{l|}{project} &
\multicolumn{1}{l|}{course} & \multicolumn{1}{l|}{other} & \multicolumn{1}{l|}{student} &
\multicolumn{1}{l|}{faculty} & \multicolumn{1}{l|}{dpt.} &
\multicolumn{1}{l|}{staff} \\ \hline project & 6 & 0 & 5 & 8 & 2 & 27 & 203 \\ \hline course & 3 & 29 & 7 & 22 & 14 & 57 & 332 \\ \hline other & 45 & 71 & 106 & 85 & 67 & 256 & 1250 \\ \hline
student & 5 & 4 & 39 & 73 & 47 & 87 & 565 \\ \hline
faculty & 1 & 2 & 8 & 21 & 48 & 38 & 443 \\ \hline
dpt. & 0 & 0 & 1 & 3 & 1 & 38 & 46 \\ \hline
staff & 0 & 0 & 1 & 2 & 3 & 8 & 53 \\ \hline
\end{tabular}
\caption{Resultate für GZip Kompressionsstufe 1}
\label{tbl:GzipL1}
\end{table}
 
\end{itemize}
\end{frame}

\begin{frame}[c]
\myframetitle{Task4}{Evaluation von GZip}
\begin{itemize}
  \fatitem{Klassifikationsmatrix für GZip(Kompressionsstufe 2)}
  \begin{table}[htbp]
\caption{Resultate für GZip Kompressionsstufe 2}
\begin{tabular}{|l|r|r|r|r|r|r|r|}
\hline
train$\downarrow$ test$\rightarrow$& \multicolumn{1}{l|}{project} & \multicolumn{1}{l|}{course} &
\multicolumn{1}{l|}{other} & \multicolumn{1}{l|}{student} &
\multicolumn{1}{l|}{faculty} & \multicolumn{1}{l|}{dept.} &
\multicolumn{1}{l|}{staff} \\ \hline project & 55 & 4 & 0 & 38 & 27 & 0 & 127 \\ \hline course & 105 & 6 & 0 & 73 & 44 & 0 & 236 \\ \hline other & 356 & 64 & 0 & 308 & 180 & 0 & 972 \\ \hline
student & 174 & 25 & 0 & 121 & 78 & 0 & 422 \\ \hline
faculty & 115 & 15 & 0 & 77 & 65 & 0 & 289 \\ \hline
dept. & 13 & 0 & 0 & 18 & 13 & 0 & 45 \\ \hline
staff & 18 & 2 & 0 & 10 & 5 & 0 & 32 \\ \hline
\end{tabular}
\label{tbl:GzipL2}
\end{table}
  
   
\end{itemize}
\end{frame}

\begin{frame}[c]
\myframetitle{Task4}{Evaluation von GZip}
\begin{itemize}
  \fatitem{Klassifikationsmatrix für GZip(Kompressionsstufe 6)}
  \begin{table}[htbp]
\caption{Resultate für GZip Kompressionsstufe 6}
\begin{tabular}{|l|r|r|r|r|r|r|r|}
\hline
train$\downarrow$ test$\rightarrow$& \multicolumn{1}{l|}{project} & \multicolumn{1}{l|}{course} &
\multicolumn{1}{l|}{other} & \multicolumn{1}{l|}{student} &
\multicolumn{1}{l|}{faculty} & \multicolumn{1}{l|}{dept.} &
\multicolumn{1}{l|}{staff} \\ \hline project & 25 & 0 & 19 & 8 & 6 & 3 & 190 \\ \hline course & 16 & 88 & 19 & 27 & 12 & 6 & 296 \\ \hline other & 136 & 109 & 125 & 112 & 53 & 60 & 1285 \\ \hline
student & 17 & 1 & 88 & 106 & 38 & 4 & 566 \\ \hline
faculty & 2 & 2 & 13 & 19 & 56 & 2 & 467 \\ \hline
dept. & 0 & 0 & 10 & 1 & 1 & 12 & 65 \\ \hline
staff & 1 & 0 & 5 & 5 & 3 & 1 & 52 \\ \hline
\end{tabular}
\label{tbl:GzipL6}
\end{table}
  
  
   
\end{itemize}
\end{frame}

\begin{frame}[c]
\myframetitle{Task4}{Evaluation von GZip}
\begin{itemize}
  \fatitem{Klassifikationsmatrix für GZip(Kompressionsstufe 9)}
  \begin{table}[htbp]
\caption{Resultate für GZip Kompressionsstufe 9}
\begin{tabular}{|l|r|r|r|r|r|r|r|}
\hline
train$\downarrow$ test$\rightarrow$& \multicolumn{1}{l|}{project} & \multicolumn{1}{l|}{course} &
\multicolumn{1}{l|}{other} & \multicolumn{1}{l|}{student} &
\multicolumn{1}{l|}{faculty} & \multicolumn{1}{l|}{dept.} &
\multicolumn{1}{l|}{staff} \\ \hline project & 28 & 0 & 35 & 10 & 2 & 4 & 172 \\ \hline course & 18 & 91 & 31 & 24 & 10 & 7 & 283 \\ \hline other & 153 & 114 & 187 & 102 & 41 & 57 & 1226 \\ \hline
student & 22 & 1 & 126 & 108 & 37 & 5 & 521 \\ \hline
faculty & 6 & 3 & 37 & 23 & 57 & 2 & 433 \\ \hline
dept. & 0 & 0 & 11 & 2 & 1 & 10 & 65 \\ \hline
staff & 1 & 0 & 6 & 5 & 3 & 1 & 51 \\ \hline
\end{tabular}
\label{tbl:GzipL9}
\end{table}
   
\end{itemize}
\end{frame}

\begin{frame}[c]
\myframetitle{Task4}{Recall, Precision, Accuracy von Compress}
\begin{itemize}
  \fatitem{Auswertung der Klassifikationsmatrizen ergibt}
\begin{table}[htbp]
\begin{tabular}{|l|r|r|r|r|}
\hline
\textbf{Test} & \textbf{Gzip(1)} & \textbf{Gzip(2)} &
\textbf{Gzip(6)} & \textbf{Gzip(9)} \\ \hline Accuracy & 8,54\% & 6,75\% & 11,23\% & 12,88\% \\ \hline
\multicolumn{5}{|l|}{\textbf{Recall}}  \\ \hline micro & 18,99\% & 15,63\% &
23,55\% & 26,10\% \\ \hline macro & 21,30\% & 14,79\% & 24,36\% & 25,27\% \\ \hline
\multicolumn{5}{|l|}{\textbf{Precision}}  \\ \hline micro & 49,51\% & 5,27\% &
28,19\% & 28,00\% \\ \hline macro & 45,56\% & 4,98\% & 28,38\% & 27,32\% \\ \hline
\end{tabular}
\caption{Auswertung der Klassifikationsresultate für Compress}
\label{tbl:GzipAccuResults}
\end{table}
  
\end{itemize}
\end{frame}

\begin{frame}[c]
\myframetitle{Task4}{Interpretation der Ergebnisse}
\begin{itemize}
  \item Die Ergebnisse sind extrem schlecht. 
  \item Raten der größten Gruppe(other) würde deutlich bessere Ergebnisse
  liefern.
  \item Aber: Je höher die Kompression desto besser werden die Ergebnisse
  \item Stellt sich die Frage: Warum sind die Ergebnisse so schlecht?
  \begin{enumerate}
  \item WebKB besteht aus sehr kleinen HTML-Seiten
  \item Viele der HTML-Seiten haben den gleichen Aufbau
  (<html><head>\ldots</body></html>)
\end{enumerate}
\item $\rightarrow$ Viele der Seiten ähneln sich stark.
\item Daher ist das klassifizieren schwierig
\end{itemize}
\end{frame}

\begin{frame}[c]
\myframetitle{Task 4}{Verbesserung der Ergebnisse}
\begin{itemize}
  \item Man erhält vermutlich bessere Ergebnisse wenn man
  \item Das HTML entfernt
  \begin{itemize}
  \item So vermeidet man eine ``Klassifikation nach HTML-Elementen''
  \item Die Ähnlichkeit zwischen den Dokumenten wird verringert
\end{itemize}
\item Einen stärker komprimierenden Algorithmus wählt
\begin{itemize}
  \item Dann werden die Dateigrößenunterschiede höher
  \item Allerdings erfordert eine stärkere Kompression mehr Rechenzeit
\end{itemize}
\end{itemize}
\end{frame}

\begin{frame}[c]
\myframetitle{Task 4.4}{Interdomain-Klassifikation}
\begin{itemize}
  \fatitem{Auf dem WebKB-Datensatz sollen die Mini-Newsgroups-Postings
  klassifiziert werden.}
  \item Für die Klassen gilt folgende Zuordnung:
  \begin{itemize}
  \item student
  \begin{itemize}
  \item misc.forsale, rec.autos, 
  rec.motorcycles, rec.sport.baseball, rec.sport.hockey
\end{itemize}
  \item project
  \begin{itemize}
  \item sci.space, sci.electronics, sci.med, sci.space
\end{itemize}
  \item other
  \begin{itemize}
  \item alt.aheism, comp.graphics, comp.os.ms-windows.misc,
  comp.sys.ibm.pc.hardware, comp.sys.mac.hardware, comp.windows.x, soc.religion.christian, talk.politics.guns, talk.politics.mideast, talk.politics.misc
\end{itemize}
\end{itemize}
\end{itemize}
\end{frame}

\begin{frame}[c]
\myframetitle{Task 4.4}{Interdomain-Klassifikation durch Compress}
\begin{itemize}
  \fatitem{Ergebnisse der Interdomain-Klassifikation}
  \item Precision: Macro 34,50\%, Micro 48,68\%
  \item Recall: Macro 31,12\%, Micro 43,30\%
  \item Accuracy: 27,50\%
\begin{table}[htbp]
\begin{tabular}{|l|r|r|r|r|r|r|r|}
\hline
train test & \multicolumn{1}{l|}{project} & \multicolumn{1}{l|}{course} & \multicolumn{1}{l|}{other} & \multicolumn{1}{l|}{student} & \multicolumn{1}{l|}{faculty} & \multicolumn{1}{l|}{dpt} & \multicolumn{1}{l|}{staff} \\ \hline
project & 24 & 0 & 58 & 94 & 15 & 5 & 50 \\ \hline
course & 11 & 0 & 173 & 190 & 39 & 17 & 29 \\ \hline
other & 141 & 3 & 471 & 460 & 177 & 55 & 568 \\ \hline
student & 7 & 0 & 60 & 436 & 49 & 41 & 222 \\ \hline
faculty & 12 & 0 & 89 & 232 & 123 & 9 & 91 \\ \hline
dpt. & 3 & 0 & 4 & 26 & 0 & 50 & 5 \\ \hline
staff & 0 & 0 & 6 & 25 & 8 & 0 & 23 \\ \hline
\end{tabular}
\caption{Ergebnismatrix der Interdomain-Klassifikation}
\label{tbl:InterdomainCls}
\end{table}
\end{itemize}
\end{frame}

\begin{frame}[c]
\myframetitle{Task 4.4}{Interdomain-Klassifikation}
\begin{itemize}
  \fatitem{Ergebnisse der Interdomain-Klassifikation}
  \item Die Interdomain-Klassifikation wirft Schiwerigkeiten auf
  \begin{itemize}
  \item Zuordnung der Klassen von einer Domain zur anderen
  \begin{itemize}
  \item Erfolgte mehr oder weniger willkürlich durch ``Bauchgefühl''
  \item Verfahren zur schematischen Zuordnung der Klassen von Domain A nach Domain B notwendig
\end{itemize}
  \item Anpassng des Datenformates der Domains
  \begin{itemize}
  \item In unserem Beispiel wurde Plaintext gegen HTML klassifiziert
  \item Lösung: Das HTML aus den WebKB-Dokumenten entfernen
\end{itemize}
  
\end{itemize}
\fatitem{$\rightarrow$ Die Interdomain-Klassifizierung es eine Notlösung, wenn ein Training nicht in Frage kommt}
  \end{itemize}
\end{frame}
