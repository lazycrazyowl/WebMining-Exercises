\begin{frame}[c]
  \myframetitle{Ranking-Verfahren}{Im Vergleich}

  \begin{itemize}
  \fatitem{Alle Ranking-Verfahren eignen sich für einen Vergleich der
  Suchergebnisse}
  \item Das tatsächlich beste Verfahren muss Anhand der Anforderungen gewählt
  werden
  \item Jen nach dem, auf welche Anforderungen die Suchmaschine optimiert werden
  soll, eignen sich die Ranking-Verfahren unterschiedlich gut.
  \begin{itemize}
    \item \textit{Precision} und \textit{Recall} ermöglichen ein Optimieren der
    Suchmaschine auf False-Positive und False-Negative-Fehlerraten. Die Qualität
    des Rankings kann nicht gemessen werden
    \item \textit{NDCG} ermöglicht ein Optimieren der Suchmaschine falls nur
    wenige Suchergebnisse gewünscht sind. Je höher der Rang, desto geringer ist der
    Einfluss eines Ergebnisdokuemnts auf das Messergebniss
    \item Die \textit{Kendall-Tau-Distanz} ermöglicht das Bestimmen der Güte
    der Sortierung der Ebnissdokumente. Es werden dabei aber keinerlei Aussagen
    über die Fehlerrate getroffen
  \end{itemize}
\end{itemize}
\end{frame}

\begin{frame}[c]
  \myframetitle{Suchverfahren im Vergleich}{HITs vs. Page-Rank}
  \begin{itemize}
  \item HITs hat einen höheren Recall bei einer geringeren Precision
  $\rightarrow$ HITs liefert mehr Ergebnisdokumente zurück.
\item   Die \textit{normalized discountent cumulative Gain} und
\textit{average Precision} haben ähnliche Werte. 
  \item Allerdings ist die \textit{normalisierte Kendall-Tau-Distanz} des
  HITs-Ergebnissets höher. Das bedeutet, PageRank sortiert die Ergebnisse
  besser.
  \fatitem{$\Rightarrow$ HITs liefert mehr gewünschte Ergebnisse während bei
  Page-Rank die Sortierung der Ergebnisse besser ist}
\end{itemize}

  
\end{frame}